\section{Rekursion und Iteration \verweis{9.8}}
	\subsection{Unterschied von Rekursion und Iteration \verweis{9.8.1}}
		\begin{compactitem}
			\item Rekursion: Die Funktion enthält Abschnitte, in der sie selbst direkt oder indirekt wieder	aufgerufen wird.
			\item Iteration: Ein Algorithmus enthält Abschnitte, die innerhalb einer Ausführung mehrfach durchlaufen werden (Schleife).
			\item Jeder rekursive Algorithmus kann auch iterativ formuliert werden.
			\item Die rekursive Form kann eleganter sein, ist aber praktisch immer ineffizienter als die iterative Form.
			\item Das Abbruchkriterium ist bei beiden Formen zentral.
		\end{compactitem}
		
	\subsection{Anwendung von rekursiven Funktionen \verweis{9.8.1}}
		\begin{compactitem}
			\item Wachstums-Vorgänge 
			\item Backtracking-Algorithmen:	z.B. Finden eines Weges durch ein Labyrinth (zurück aus Sackgasse und neuen	Weg prüfen)
			\item Traversierungen von Suchbäumen
			\item In Mathematik rekursiv definierte Algorithmen
		\end{compactitem}	
	
	\subsection{Beispiel anhand der Fakultätsberechnung \verweis{9.8.2}}
		\begin{minipage}[t]{9 cm}
			\subsubsection{Iterativ}	
				\lstinputlisting[language=C,tabsize=2]{code/rekursion_fakultaet_iterativ.c}
		\end{minipage}
		\hspace*{0.5cm}
		\begin{minipage}[t]{9 cm}
			\subsubsection{Rekursiv}
				\lstinputlisting[language=C,tabsize=2]{code/rekursion_fakultaet_rekursiv.c}
		\end{minipage}
		
	\subsection{Beispiel anhand der Binärdarstellung \verweis{9.8.3}}
			\begin{minipage}[t]{9 cm}
				\subsubsection{Iterativ}	
					\lstinputlisting[language=C,tabsize=2]{code/rekursion_binaer_iterativ.c}
			\end{minipage}
			\hspace*{0.5cm}
			\begin{minipage}[t]{9 cm}
				\subsubsection{Rekursiv}
					\lstinputlisting[language=C,tabsize=2]{code/rekursion_binaer_rekursiv.c}
			\end{minipage}		